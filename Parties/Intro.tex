\chapter{Information}

\section{Définition d'un jeux société}

\paragraph{Le jeu de société est un jeu qui se pratique à plusieurs personnes, par opposition aux jeux qui se pratiquent seul, les jeux solitaires ou casse-tête. }~\\

Un jeu est une activité de loisir soumise à des règles qui définissent les moyens, les contraintes et les objectifs à atteindre au cours de la partie. La finalité de cette activité est le divertissement que les participants en retirent en essayant de remporter la partie.\\

On différencie généralement les jeux de société des jeux vidéo ainsi que des activités essentiellement physiques, qu’on appelle plus volontiers « sports », même si la limite entre les sports et les jeux de société est difficile à déterminer précisément.\\

Les jeux de société se caractérisent par :

\begin{itemize}
	\item un règlement — la règle du jeu — plus ou moins complexe.
	\item un nombre de participants (au minimum deux) variable, mais le plus souvent limité à quelques personnes, bien qu’il existe certains jeux avec des variantes pour jouer seul ou en grand groupe.
	\item la plupart du temps, l’existence d’un ou de plusieurs supports matériels : tablier (souvent appelé « plateau »), cartes, aire de jeu, dés, pions, feuilles de papier, stylos…
\end{itemize}~\\

Les jeux de société sont généralement considérés comme des jeux intellectuels faisant souvent appel à la réflexion, mais ils peuvent aussi faire appel au hasard, à la mémoire, à l’adresse, à l’observation, à la vivacité, au bon sens… On note une tendance récente, depuis les années 1990, à des jeux plus conviviaux dans lesquels le but essentiel est de créer une bonne ambiance.\\

Le terme « jeu de société » regroupe ainsi tous les jeux (non sportifs et non vidéo) rassemblant au moins deux joueurs mais ce terme est parfois employé à tort dans un sens plus restrictif pour désigner les jeux qui ne font pas partie des grands classiques — échecs, bridge — ou des jeux qui n’appartiennent pas à des branches spécialisées comme les jeux de guerre, les jeux de figurines, les jeux de rôle ou les jeux de cartes à collectionner. 

\newpage

\section{Définition Règle de jeux}

\paragraph{Ensemble de consignes établies pour conditionner le bon déroulement d'un jeu. Elles instaurent les facultés et les contraintes qui se présentent à chaque joueur et doivent être énoncées clairement à chacun d'entre eux avant le début du jeu.}~\\

Un jeu est déterminé par trois éléments :
	\begin{itemize}
		\item la situation de départ
		\item le but à atteindre
		\item les règles : possibilités et contraintes qui doivent être respectées par les joueurs
	\end{itemize}~\\

	On appelle règle de jeu l'ensemble des principes qui régissent les conditions de déroulement d'un jeu jusqu'à la victoire : \\
		\begin{itemize}
		\item Les règles des jeux traditionnels sont transmises par oral ou par écrit
		\item Les règles des jeux contemporains sont généralement rédigées par l'auteur ou par l'éditeur du jeu
	\end{itemize}~\\

	En fonction du type de jeu (jeu de société, jeu de rôle, jeu d'extérieur, etc.), la règle comprend différents chapitres (préambule, matériel, etc.)\\


 